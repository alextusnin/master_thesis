\intro
С момента изобретения первого лазера~\cite{maiman1960stimulated}, интерес к изучению взаимодействия света с веществом и его использованию в новых технологиях резко вырос, что привело к появлению фотоники как самостоятельной дисциплины~\cite{lifante2003integrated}. Фотоника включает в себе широкий спектр задач, включая проблемы
 телекоммуникации~\cite{agrawal2012fiber},
 ближнепольной микроскопии~\cite{betzig1992near},
 метрологии~\cite{kim2009metrology},
и многие другие~\cite{saleh2019fundamentals}. Во всех этих областях идёт разработка и исследование интегрированных оптических устройств, в которых осуществляется связь между оптическим полем и его окружением.  Одним из типов такой связи являются опто-механические взаимодействия. 

Использования света для манипуляции микро-объектами находит применение в Micro-Opto-Electro-Mechanical-Systems (MOEMS) технологиях~\cite{motamedi1994micro}, изучении биологических и макромолеклярных систем~\cite{Ashkin1517,grier2003revolution}. Оптические ловушки также используются для изучения законов термодинамики для малых статистических ансамблей~\cite{wang2002experimental}. Оптомеханические эффекты использовались также для охлаждения механических систем (в работе~\cite{schliesser2006radiation} авторы смогли охладить механический осциллятор с комнатной температуры до $11$ К) и для получения <<сжатых>> оптических полей~\cite{fabre1994quantum}, которые используются для увеличения чувствительности детекторов гравитационных волн~\cite{aasi2013enhanced}.

Одним из самых распространённых методов оптического воздействия на механическое движение является использование градиентных  сил в сильносфокусированных оптических полях~\cite{ashkin1986observation}. Такие устройства называются лазерными пинцетами, и они позволяют двигать нейтральные частицы вдоль градиента интенсивности пучка, благодаря взаимодействию наведённого дипольного момента в частицах с градиентом внешнего поля, а также давлению света на поверхность. В случае сильной фокусировки преобладает сила, вызванная градиентом, поэтому эти силы также называют градиентными.

Когда давление света на поверхность превалирует над другими силами, происходит передача импульса при отражении от фотонов к частице, что также позволяет проводить оптомеханические манипуляции. В работе~\cite{sukhov2011negative} авторы продемонстрировали, что специально подобранный оптический пучок может вызывать силы действующие на частицы как вдоль, так и против распространения пучка, причём форма и структура частиц могут быть произвольными. Однако существует ещё один тип оптомеханических взаимодействий, который вызван взаимодействием поверхностных плазмонов, которому и посвящена эта работа.

Плазмон --- это электромагнитное (ЭМ) возбуждение, связанное со свободными носителями заряда. Поверхностный плазмон-поляритон (далее --- поверхностный плазмон) --- это связанное состояние ЭМ излучения с поверхностными электронами в металле на границе металл-диэлектрик. Несмотря на свою квантовую природу, уравнения Максвелла хорошо опимывают такие явления. Одной из отличительных особенностей поверхностных плазмонов является локальное увеличение амплитуды поля вблизи поверхности на субволновых масштабах, что может быть использовано для миниатюризации фотонных устройств~\cite{barnes2003surface}. Плазмоны использовались для обнаружения рамановского рассеяния на одиночных молекулах~\cite{PhysRevLett.78.1667}, ускорении нерелятивистских электронов на диэлектрических подложках~\cite{breuer2013laser}.
С помощью плазмонов удалось разработать нано-оптические твизеры, которые стали широкоиспользуемыми для оптических ловушек~\cite{juan2011plasmon,xu2002surface}. Но во всех этих работах плазмоны использовались как средство для локального усиления полей.

Новый тип оптических сил был представлен в работах~\cite{roels2009tunable,li2009tunable,dholakia2010colloquium,nesterov2011light,Frumin11} для различных геометрий. В работах~\cite{nesterov2011light,Frumin11} авторы показали, что в условиях возбуждения плазмонного резонанса на границе металл-диэлектрик (при падении TM поляризованного излучения)~\cite{zayats2005nano} между близко расположенными параллельными металлическими пластинами возникает сила притяжения, амплитуда которой (для золотых пластин) много больше, чем сила отталкивания для поляризации TE, когда поверхностные плазмоны не возбуждаются. В связи с этим эту силу назвали плазмонной. 

Аналитическое выражение для плазмонной силы и его анализ для субволновой щели был представлен в работах~\cite{nesterov2011light,Frumin11}, однако в них авторы предполагали, что в щели возбуждается лишь одна фундаментальная мода. В работе~\cite{Shapiro16} авторы учли возбуждение высших мод при нормальном падении плоской волны на щель при произвольной ширине. Авторы показали, что одномодовое приближение хорошо работает для субволновой щели, но увеличение расстояния между пластинами приводит к рождению новых распространяющихся мод. Однако расчёт полей был произведён в приближении идеального проводника\footnote{Под идеальным проводником 
понимается случай, когда проводимость вещества равна бесконечности. Таким образом, электрическое поле внутри проводника равно нулю, что приводит к непрерывности касательной компоненты электрического поля ($E_{\tau} = 0$) и непрерывности нормальной производной касательной компоненты магнитного поля 
$\frac{\partial H_{\tau}}{\partial n} = 0$}, что не позволяет рассчитать плазмонную силу для произвольной ширины щели.


В данный момент в Физико-Техническом Институте, г. Брауншвейг (Германия) ведётся эксперимент по измерению плазмонной силы. Первые результаты~\cite{nies2018direct} были получены для ширины, сопоставимой с длиной волны, так как поверхность золота, используемого в качестве проводника, имеет неровности порядка длины волны, из-за чего нельзя сблизить пластины на субволновые расстояния. В связи с этим возникла необходимость учёта экспериментальных неточностей.  Таким образом, цель данной работы --- это построение теории для плазмонной силы, которая справедлива для произвольной ширины щели, учитывает эффекты наклонного падения возбуждающей волны и фокусировки пучка. 

Работа состоит из двух глав, трёх приложений и списка литературы.
В первой главе рассматривается общая теория для плазмонной силы в щели. Во второй главе представлены численные результаты с учётом конечного количества возбуждаемых мод, а также численное сравнение теории с моделированием в среде COMSOL. В приложении А представлен анализ собственных значений для мод в щели и предельный переход к идеальному проводнику; в приложении Б рассмотрены методы моделирования бесконечных границ в численных методах; в приложении В описан эксперимент по измерению плазмонной силы. 

% Результаты работы были представлены на конференциях:
% \begin{itemize} 
%     \item Туснин А.К., Фрумин Л.Л., Белай О.В., Шапиро Д.А. Влияние
% юстировки на плазмонную силу в субволновой щели // Материалы 8-го Российского семинара по волоконным лазерам (Новосибирск, 2-7 сентября 2018) / Новосибирск: ИАиЭ СО РАН, 2018,
% C. 35—36. DOI: 10.31868/RFL2018.35-36
%     \item Туснин А. К. Электрическое поле в щели между проводящими
% плоскостями // Материалы 55-й Международной научной студенческой конференции МНСК-2017: Фотоника и квантовые оптические технологии (Новосибирск, 17–20 апреля 2017) / Новосиб. гос.
% ун-т., Новосибирск, 2017. - C.36;
% \end{itemize}
% и опубликованы в:
% \begin{itemize} 
%     \item Frumin L., Tusnin A., Belai O., Shapiro D. Effects of imperfect
%     angular adjustment on plasmonic force // Opt. Express. –
%     2017. – V.25, No25. – P. 31801-31809. DOI: 10.1364/OE.25.031801
%     (arxiv.org:1707.08688, 22 Jul. 2017);
%     \item Tusnin А., Shapiro D. Influence of higher modes on plasmonic force
% in narrow slit // OSA Continuum. – 2018. – V.1, No 2. – P. 384-391.
% DOI: 10.1364/OSAC.1.000384
% \end{itemize}

