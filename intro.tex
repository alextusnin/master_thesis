\intro

  Особенности взаимодействия света с веществом открывают новые возможности в развитии технологий и изучения природы. Благодаря тому, что свет может вызывать как консервативные так и неконсервативные силы, действующие на окружение, в настоящее время появились различные методики для манипуляции нейтральными частицами нано- и микро-размеров, что может быть использовано в широком спектре прикладных задач. 

\subsection{Консервативные силы}  
В настоящее время наибольшее распространение получили методики, основанные на консервативных силах. Например, в работе~\cite{grier2003revolution} рассмотрены примеры использования оптических пинцетов в сильно сфокусированных лазерных полях. 
Подобные лазерные пинцеты позволяют двигать частицы вдоль градиента интенсивности пучка благодаря взаимодействию наведённого дипольного момента в частицах с градиентом внешнего поля, а так же давлению света на поверхность. В случае сильной фокусировки преобладает давление, вызванное градиентом, поэтому эти силы называют консервативными. Возникающие силы (порядка $100$ пН) стали широко используемыми в изучении биологических и макромолекулярных системах. Возможность неинвазивного собирания частиц позволило производить новые эксперименты в области классической статистической механики. 

\subsection{Неконсервативные силы}


\subsection{Плазмонные силы}
Данная работа посвящена изучению силы давления света между двумя металлическими пластинами. Возникающая силы обусловлена взаимодействием поверхностных плазмонов на рассматриваемых металлических поверхностях, откуда и происходит название <<плазмонная сила>>. 

Впервые аналитическое выражение для светового давления в субволновой щели было получено в работе~\cite{Frumin11} в приближении, что в щели возбуждается лишь одна мода, амплитуда которой равняется амплитуде падающей волны на щель и которая не зависит от ширины щели. Полученная оценка в $0.36$ нН, что больше, чем градиентные силы.  В работе~\cite{Shapiro16}
авторы рассчитали амплитуду поля в щели в зависимости от её ширины для идеального проводника\footnote{В данной работе под идеальным проводником 
понимается случай, когда проводимость вещества равна бесконечности. Таким образом, поле внутри проводника равно нулю, что приводит к непрерывности касательной компоненты электрического поля ($E_{\tau} = 0$) и непрерывности нормальной производной касательной компоненты магнитного поля 
$\frac{\partial H_{\tau}}{\partial n} = 0$} в случае нормального падения на щель. 

В данный момент в Физико-Техническом Институте, г. Брауншвейг (Германия) ведётся эксперимент по измерению плазмонной силы, в связи с чем возникла 
необходимость в построении теории рассматриваемой силы, которая будет учитывать возможные неточности эксперимента. Таким образом, цель данной работы заключается в расчёте плазмонной силы для реального металла в случае произвольной ширины, учёт наклонного падения возбуждающей волны, конечные 
размеры пучка. 
