\conclusion
В ходе проделанной работы была разработана теория светоиндуцированной плазмонной силы в щели между двумя металлическими пластинами. Для этого было найдено распределение электромагнитного поля в двумерной геометрии при падении излучения с произвольным фронтом. В частности, было изучено влияния наклонного падения плоской волны на щель, вклад высших гармоник в величину силы при нормальном падении для плоского фронта и для цилиндрического гауссова пучка. 

В ходе расчётов было получено, что при малых углах падения $\gamma$ поправка к силе порядка $\gamma$, однако с ростом угла падения сила отталкивания, вызванная первой нечётной модой, начинает превалировать над фундаментальной модой, что может привести к отталкиванию. При расчёте вклада высших мод было получено, что для силы достаточно учитывать лишь вклад распространяющихся мод. Рождение каждой новой моды приводит к резкому изменению силы. Фокусировка пучка приводит к усилению возбуждения высших мод при увеличении ширины, что приводит в свою очередь к уменьшению величины силы. Учёт конечной проницаемости приводит к разнице не больше $10\%$ (при значении $\eps = -100 + 10i$) по сравнению с идеальным проводником при различных ширинах щелей, кроме областей вблизи рождения новых мод. С другой стороны, различные  значения проницаемости сильно влияют на константы распространения, которые прямым образом входят в выражение для силы. Например, для $\eps = -100 + 10 i$ сила вблизи порога рождения второй чётной моды резко уменьшается, но остаётся положительной, однако для $\eps = -360 + 60 i$ сила становится расталкивающей, а с увеличением ширины вновь становится силой притяжения. Это приводит к возникновению двух точек, где сила обращается в нуль, одна из которых является точкой устойчивого равновесия. 

С помощью программного пакета COMSOL Multyphysics было проведено численное моделирование методом конечных элементов задачи в случае нормального падения плоской волны. Результаты моделирования хорошо воспроизводят амплитуды полей и константы распространения, что свидетельствует о хорошем соответствии теории и численного моделирования.