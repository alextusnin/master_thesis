\documentclass[%
master,      % тип документа
natbib,      % использовать пакет natbib для "сжатия" цитирований
%subf,        % использовать пакет subcaption для вложенной нумерации рисунков
%href,        % использовать пакет hyperref для создания гиперссылок
colorlinks,  % цветные гиперссылки
%fixint,     % включить прямые знаки интегралов
]{disser}
\usepackage[
  a4paper, mag=1000,
  %left=2.5cm, right=1cm, top=2cm, bottom=2cm, headsep=0.7cm, footskip=1cm
  left=3cm, right=1.5cm, top=2cm, bottom=2cm, headsep=0.7cm, footskip=1cm
]{geometry}
\usepackage[hyphens]{url}%для переноса строки в ссылках
\usepackage{hyperref}


\usepackage{pdfpages}
\usepackage[intlimits]{amsmath}
\usepackage{amssymb,amsfonts}

\usepackage{mathtext}
\usepackage[T2A]{fontenc}
%\usepackage{ucs}
\usepackage[utf8]{inputenc}  % или какая у тебя там входная кодировка
\usepackage[russian]{babel}
%\usepackage[
%backend=biber,
%style=numeric,
%sorting=ynt
%]{biblatex}
%\usepackage[utf8]{inputenc}
\linespread{1.3}   %one half line spacing
%\ifpdf\usepackage{epstopdf}\fi
\usepackage[autostyle]{csquotes}
\usepackage{appendix}

\usepackage[intoc,nocfg,russian]{nomencl}
\newcommand{\nomencl}[2]{#1 --- #2\nomenclature{#1}{#2}}
\setlength{\nomlabelwidth}{3em}
\setlength{\nomitemsep}{-\parsep}
\renewcommand{\nomlabel}[1]{#1 ---}
\makenomenclature

\usepackage{subcaption}
\usepackage{tikz,graphicx}
\usetikzlibrary{decorations.pathmorphing, patterns,shapes}

\newcommand{\eps}{\varepsilon}
\newcommand{\sinc}{\mathrm{sinc}\;}
\renewcommand{\vec}[1]{\mathbf{#1}}
\renewcommand{\Re}{\textrm{Re}\,}
\renewcommand{\Im}{\textrm{Im}\,}

\usepackage{mathtools}
\DeclarePairedDelimiter\abs{\lvert}{\rvert}%
\DeclarePairedDelimiter\norm{\lVert}{\rVert}

% Плавающие рисунки "в оборку".
\usepackage{wrapfig}


\renewcommand{\epsilon}{\varepsilon}
\renewcommand{\kappa}{\varkappa}
\renewcommand{\phi}{\varphi}
\newcommand{\diff}{\mathrm{d}}



\graphicspath{{./figures/}}
% Включать подсекции в оглавление
\setcounter{tocdepth}{2}
